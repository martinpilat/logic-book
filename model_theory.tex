%!TEX root = main.tex

\chapter{Basics of model theory and decidability}

In the last part of the lecture, we will discuss the basics of model theory, decidability of theories and the incompleteness of some theories. Model theory is a rather modern branch of logic that rapidly developed in the 1990s. As its name suggests, it studies the models of theories and answers (among others) questions like: what is the number of models of a theory $T$ up to isomorphisms, is the theory $T$ complete, etc.

The decidability of theories deals with the problem if we can algorithmically decide whether a sentence is provable in a theory $T$ or not. If it is, the theory is called decidable. 

Finally, we will discuss the so called Gödel incompleteness theorems that state that once a theory is sufficiently strong, it is incomplete. We will also show that truth cannot be defined in logic.

\section{Model Theory}

We have actually already seen some terms that are considered a part of model theory. For example, we know that two structures are elementarily equivalent if they satisfy the same sentences. We have also defined the theory of a structure $\cA$ -- $\Th(\cA)$ as the set of all sentences valid in $\cA$. This set is also a theory (it is a set of formulas). Moreover, $\Th(\cA)$ is always a complete theory, if $\cA \vDash T$, then $\Th(\cA)$ is a simple (complete) extension of $T$, and if $\cA \vDash T$ and $T$ is complete, then $\Th(\cA)$ is equivalent to $T$, i.e. $\Th(\cA) = \theta^L(T)$. We can also easily see that for every models $\cA, \cB$ of a theory $T$, $\cA \equiv \cB$ if and only if $\Th(\cA)$ and $\Th(\cB)$ are equivalent theories.

Before we discuss some of the ideas of model theory further, we define some common algebraic theories. 
\begin{enumerate}
  \item the \emph{theory of groups} in the language $L = \struct{+, -, 0}$ with equality has axioms 
  	\begin{align*}
  		& x + (y + z) = (x + y) + z \\
  		& 0 + x = x = x+0 \\
  		& x + (-x) = 0 = (-x) + x
  	\end{align*}
  \item the \emph{theory of Abelian groups} has moreover the axiom $x + y = y+x$
  \item the \emph{theory of rings} in $L = \struct{+, -, \cdot, 0, 1}$ has moreover axioms 
    \begin{align*}
    	& 1 \cdot x = x = x \cdot 1 \\
    	& x \cdot (y \cdot z) = (x \cdot y) \cdot z \\
    	& x \cdot (y + z) = x \cdot y + x \cdot z \\
    	& (x + y) \cdot z = x\cdot z + y \cdot z
    \end{align*}
  \item the \emph{theory of commutative rings} has moreover the axiom $x \cdot y = y \cdot x$, and
  \item the \emph{theory of fields} in the same language has additional axioms 
    \begin{align*}
    	& x \neq 0 \to (\exists y)(x \cdot y = 1) \\
    	& 0 \neq 1
    \end{align*}
\end{enumerate}

Another important theory is the \emph{theory of dense linear orders} $DeLO^*$ of the language $L = \struct{\leq}$ with equality that has axioms
\begin{align*}
& x \leq x \\
& x \leq y \land y \leq x \to x = y \\
& x \leq y \land y \leq z \to x \leq z \\
& x \leq y \lor y \leq x \\
& x < y \to (\exists x)(x < z \land z < y) \\
& (\exists x)(\exists y)(x \neq y)
\end{align*}
where $x < y$ means $x \leq y \land x \neq y$. Let $\varphi$ be the sentence $(\exists x)(\forall y)(x \leq y)$ and let $\psi$ be the sentence $(\exists x)(\forall y)(y \leq x)$. We will see that $DeLO^*$ has the following four simple complete extensions (and none other): 
\begin{align*}
 DeLO &= DeLO^* \cup \{\neg \varphi, \neg \psi\} \\
 DeLO^+ &= DeLO^* \cup \{\neg \varphi, \psi\} \\
 DeLO^- &= DeLO^* \cup \{\varphi, \neg \psi\} \\
 DeLO^\pm &= DeLO^* \cup \{\varphi, \psi\}
\end{align*} 

We already know the Lövenheim-Skolem theorem -- let $T$ be a consistent theory of a countable language $L$. If $L$ is without equality, then $T$ has a countably infinite model. If $L$ is with equality, then $T$ has a model that is countable (finite or countably infinite). 

A corollary of the theorem is that for every structure $\cA$ of a countable language without equality, there exists a countably infinite structure $\cB$ such that $\cA \equiv \cB$. Why? We know that $\Th(\cA)$ is consistent since it has a model $\cA$ and by the Lövenheim-Skolem theorem it has a countably infinite model $\cB$. Since $\Th(\cA)$ is complete $\cA$ must be elementarily equivalent to $\cB$.

Similarly for theories in language with equality, but we additionally need the assumption that $\cA$ is infinite: for every infinite structure $\cA$ of a countable language without equality, there exists a countably infinite structure $\cB$ such that $\cA \equiv \cB$. The proof is similar to the one above. This time we also know that the sentence ``there are exactly $n$ elements'' is not valid in $\cA$ for any $n$ and therefore also not in $\cB$ and therefore $\cB$ must be infinite.

These corollaries are quite strong, as is demonstrated in the following proof of existence of a countable algebraically closed field. We say that a field $\cA$ is \emph{algebraically closed} if every polynomial (of non-zero degree) has a root in $\cA$, i.e. we have for each $n \in \Nat$ $$\cA \vDash (\forall x_{n-1})\dots(\forall x_0)(\exists y)(y^n + x_{n-1}\cdot y^{n-1} + \dots + x_1\cdot y + x_0 = 0)\,.$$

For example, the field $\mathbb{C} = \struct{\mathbb{C}, +, -, \cdot, 0, 1}$ of complex numbers is algebraically closed, while the fields of real numbers $\Real$ or rational numbers $\Rat$ are not. But since $\mathbb{C}$ is closed and infinite, by the previous corollary, there is a countable structure elementarily equivalent to $\mathbb{C}$ and therefore also algebraically closed.

We now want to introduce the $\omega$-categorical criterium of completeness. To this end, we first need to define what is an isomorphism of structures. Let $\cA$ and $\cB$ be structures of a language $L = \struct{\cF, \cR}$. A bijection $h: A \to B$ is a \emph{isomorphism of structures $\cA$ and $\cB$} if both $h(f^A(a_1, \dots, a_n)) = f^b(h(a_1), \dots, h(a_n))$ for every $n$-ary function symbol $f \in \cF$ and every $a_1, \dots, a_n \in A$, and $R^A(a_1, \dots, a_n) \Leftrightarrow R^B(h(a_1), \dots, h(a_n))$ for every $n$-ary relation symbol $R \in \cR$ and every $a_1, \dots, a_n \in A$. We say that $\cA$ and $\cB$ are isomorphic (via $h$) ($A \simeq B$) if there is an isomorphism $h$ of $\cA$ and $\cB$. We also say that $\cA$ is isomorphic with $\cB$. An automorphism of a structure $\cA$ is an isomorphism of $\cA$ with $\cA$. 

It is easy to show that isomorphisms preserve semantics, i.e. let $\cA$ and $\cB$ be structures of a language $L = \struct{\cF, \cR}$, a bijection $h: A \to B$ is an isomorphism of $\cA$ and $\cB$ if and only if both $h(t^A[e])=t^B[he]$ for every term $t$ and $e: \SVar \to A$, and $\cA \vDash \varphi[e] \Leftrightarrow \cB \vDash \varphi[he]$ for every formula $\varphi$ and $e: \SVar \to A$. This gives us a corollary that for every structures $\cA$ and $\cB$ of the same language $\cA \simeq \cB \Rightarrow \cA \equiv \cB$, i.e. if two structures are isomorphic, they are also elementarily equivalent. The other implication does not generally hold, as we can have two elementarily equivalent structures (e.g. the field of complex numbers $\mathbb{C}$ and the countable algebraically closed field from the previous example) that are elementarily equivalent and not isomorphic (because they have different cardinality). However, the following lemma shows that the implication holds for finite structures in a language with equality.

\begin{lemma}
For every finite structures $\cA$ and $\cB$ of a language with equality $\cA \equiv \cB \Rightarrow \cA \simeq \cB$.
\end{lemma}
\begin{proof}
First of all, we can see that $|A| = |B|$ as the sentence ``there are exactly $n$ elements can be expressed in $L$''. For a finite language $L$, we can write a formula $\varphi$ that defines $\cA$ up to isomorphism\sidenote{The sentence says ``there are exactly $n$ elements $a_1, \dots, a_n$ satisfying exactly those atomic formulas on function values and relations that are valid in the structure $\cA$''.}. This formula holds both in $\cA$ and $\cB$ since they are elementarily equivalent and therefore they are also isomorphic.

For an infinite language $L$: there is only finite number of bijections between $A$ and $B$, assume for contradiction, that none of them is an isomorphism. For each bijection $h_i$ choose a relation $R_i$ in $L$ that is not preserved in this bijection. Let $L' \subseteq L$ is the language that contains only these relations. Obviously $L'$ is finite. Then the reducts of $\cA$ and $\cB$ to $L'$ are elementarily equivalent and therefore isomorphic (by the previous paragraph) which is a contradiction.
\end{proof}

A corollary of this lemma is that if a complete theory $T$ in a language with equality has a finite model, then all models of $T$ are isomorphic.

An \emph{isomorphic spectrum} of a theory $T$ is given by the number $I(\kappa, T)$ of mutually non-isomorphic models of $T$ for every cardinality $\kappa$. A \emph{theory $T$ is $\kappa$-categorical} if it has exactly one model of cardinality $\kappa$ (up to isomorphism), i.e. $I(\kappa, T) = 1$. 

For example, the theory $DeLO$ is $\omega$-categorical. Let $\cA, \cB \vDash DeLO$, with $A = \{a_i\}_{i \in \Nat}$ and $B = \{b_i\}_{i \in \Nat}$. By induction on $n$ we can find injective $h_n \subseteq h_{n+1} \subset A \times B$ preserving the ordering such that $\{a_i\}_{i < n} \subseteq \dom(h_n)$ and $\{b_i\}_{i<n} \subseteq \rng(h_n)$. Then $\cA \simeq \cB$ via $h = \cup h_n$.

Similarly, we obtain that $\cA = \struct{\Rat, \leq}, \cA \restrict [0, 1), \cA \restrict (0, 1]$, and $\cA \restrict [0, 1]$ are up to isomorphism all countable models of $DeLO^*$. Therefore $$I(\kappa, DeLO^*) = \twopartdef{0}{\kappa \in \Nat\,,}{4}{\kappa = \omega\,.}$$

This finally leads to the $\omega$-categorical criterium of completeness (similar criteria also hold for cardinalities bigger than $\omega$):

\begin{theorem}
Let $L$ be at most countable language.
\begin{enumerate}
	\item If a theory $T$ in $L$ without equality is $\omega$-categorical, it is complete.
	\item If a theory $T$ in $L$ with equality is $\omega$-categorical and without finite models, it is complete.
\end{enumerate}
\end{theorem}
\begin{proof}
Every model of $T$ is elementarily equivalent with some countably infinite model of $T$, but such model is unique up to isomorphism, therefore all models of $T$ are elementarily equivalent and $T$ is complete.
\end{proof}

\section{Decidability}

Now, we would like to know if a given problem is algorithmically decidable (e.g. if a sentence is provable in a theory). We can formally define the notion of algorithm, for example using Turing machines. Decision problems can then be encoded into sets of natural numbers corresponding to the positive instances (with answer ``yes''). For example, $SAT = \{\lceil \varphi \rceil | \varphi \text{ is a satisfiable formula }\}$, where $\lceil \varphi \rceil$ is the natural number representing the formula $\varphi$. We say that a set \emph{$A \subseteq \Nat$ is recursive}, if there is an algorithm that for every input $x \in \Nat$ halts (stops) and correctly tells whether or not $x \in A$. We say that such algorithm \emph{decides $x \in A$}. We say that a \emph{set $A \subseteq \Nat$ is recursively enumerable} if there is an algorithm that for every input $x \in \Nat$ halts if and only if $x \in A$. We say that such algorithm \emph{recognizes $x \in A$}. Equivalently, $A$ is recursively enumerable if there is an algorithm that generates (enumerates) all elements of $A$. 

\begin{lemma}
For every $A \subseteq \Nat$ $A$ is recursive if and only if both $A$ and $\bar{A}$ recursively enumerable.
\end{lemma}
\begin{proof}
($\Rightarrow$) If $A$ is recursive, there is an algorithm that always halts and returns whether $x \in A$. We can make an algorithm that halts only if $x \in A$ (check if $x \in A$, if it is halt, otherwise enter an infinite loop). Therefore $A$ is recursively enumerable. Similarly for $\bar{A}$. 

($\Leftarrow$) If both $A$ and $\bar{A}$ are recursively enumerable, there are two algorithms $P_1$ and $P_2$ such that $P_1$ recognizes $x \in A$ and $P_2$ recognizes $x \in \bar{A}$. We can start both these algorithms in parallel (make one step of $P_1$, then one step of $P_2$, then another step of $P_1$ and so on). For each $x \in \Nat$, one of these algorithms eventually halts. If $P_1$ halts first $x \in \Nat$, otherwise $x \notin \Nat$. Therefore $A$ (and also $\bar{A}$) is recursive.
\end{proof}

In logic, we are especially interested if a given theory is algorithmically decidable. We assume a recursive language $L$. A \emph{theory $T$ of $L$ is decidable}, if $\Thm(T)$ is recursive, otherwise $T$ is \emph{undecidable}. It means, a theory is decidable, if there is an algorithm that for each sentence $\varphi$ decides whether $T \vdash \varphi$ or not. 

For every theory $T$ of $L$ with recursively enumerable axioms, $\Thm(T)$ is recursively enumerable -- the construction of systematic tableau from $T$ with root $F \varphi$ assumes a given enumeration of axioms of $T$. Since $T$ has recursively enumerable axioms, the construction provides an algorithm that recognizes $T \vdash \varphi$.

If the theory $T$ from the previous paragraph is additionally complete, then $\Thm(T)$ is recursive, i.e. $T$ is decidable. Similarly to the proof of the lemma above, we can start the construction of systematic tableaux from $T$ with roots $F \varphi$ and $T \varphi$. Since $T$ is complete, then $T \nvdash \varphi$ if and only if $T \vdash \neg \varphi$ for every sentence $\varphi$. The construction thus provides an algorithm that decides $T \vdash \varphi$.

We now know that complete theories (with recursively enumerable axioms) are decidable, however, completeness is a rather strong assumption. In fact, it is enough, if all the simple complete extensions of a theory $T$ are recursively enumerable, i.e.. if there is an algorithm $\alpha(i,j)$ that generates the $i$-th axiom of th $j$-th extension (in some enumeration) or announces that such axiom does not exist.

So, if a theory $T$ has recursively enumerable axioms and the set of all simple complete extensions of $T$ is recursively enumerable, then $T$ is decidable. By the previous observation there is an algorithm that recognizes $T \vdash \varphi$ (the construction of systematic tableau). On the other hand, if $T \nvdash \varphi$ then $T' \vdash \neg \varphi$ for some simple complete extension $T'$ of $T$. We can construct the systematic tableaux with root $T \varphi$ from all the extensions in parallel. In the $i$-th step we construct tableaux up to level $i$ for the first $i$ extensions. One of these tableaux is eventually a proof of $T' \vdash \neg \varphi$ and therefore this construction recognizes $T \nvdash \varphi$. Therefore, $T$ is decidable.

There are many theories that are decidable, although they are not complete, for example:
\begin{enumerate}
  \item the \emph{theory of pure equality} with no axioms in language $L = \struct{}$ with equality,
  \item the \emph{theory of unary predicate} with no axioms in language $L = \struct{U}$ with equality, where $U$ is a unary relation symbol,
  \item the theory of dense linear orders $DeLO^*$,
  \item the theory of algebraically closed fields in $L = \struct{+, =, \cdot, 0, 1}$ with the axioms of fields and moreover the axioms for all $n \geq 1$ $$(\forall x_{n-1})\dots(\forall x_0)(\exists y)(y^n + x_{n-1}\cdot y^{n-1} + \dots + x_1\cdot y + x_0 = 0)\,,$$
  \item the theory of Abelian groups, and
  \item the theory of Boolean algebras.
\end{enumerate}

\section{Axiomatizability}

Another interesting question is whether we can effectively (i.e. algorithmically, recursively) describe common mathematical structures. We say that a class $K \subseteq M(L)$ is \emph{recursively axiomatizable} if there is a recursive theory $T$ of language $L$ with $M(T)=K$. A \emph{theory is recursively axiomatizable} if $M(T)$ is recursively axiomatizable, i.e. if there is an equivalent recursive theory.

For example, for every finite structure $\cA$ of a finite language with equality the theory $\Th(\cA)$ is recursively axiomatizable. Thus, $\Th(\cA)$ is decidable. Let $A = \{a_1, \dots, a_n\}$, $\Th(\cA)$ can be axiomatized by a single sentence that describes $\cA$. The sentence is of the form ``there are exactly $n$ elements $a_1, \dots, a_n$ satisfying exactly those atomic formulas on function values and relations that are valid in the structure $\cA$''.

For example the following structures $\cA$ have recursively axiomatizable $\Th(\cA)$:
\begin{enumerate}
  \item $\struct{\Int, \leq}$ by the theory of discrete linear orderings, 
  \item $\struct{\Rat, \leq}$ by the theory of dense linear ordering without ends ($DeLO$),
  \item $\struct{\Nat, S, 0}$ by the theory of successor with zero, 
  \item $\struct{\Nat, S, +, 0}$ by so called Presburger arithmetic, 
  \item $\struct{\Real, +, -, \cdot, 0, 1}$ by the theory of real closed fields, and 
  \item $\struct{\mathbb{C}, +, -, \cdot, 0, 1}$ by the theory of algebraically closed fields with characteristic 0.
\end{enumerate}
By the previous observation, for all the above structures $\cA$, the theory $\Th(\cA)$ is decidable.

Apart from the recursive axiomatizability, we may also be interested if a class of structures can be axiomatized by theories with other properties. For example, a \emph{class $K \subseteq M(L)$ is openly axiomatizable}, if there is an open theory $T$, such that $M(T) = K$. A \emph{theory is openly axiomatizable}, if $M(T)$ is openly axiomatizable. We already know that if $T$ is openly axiomatizable then every substructure of its model is also its model. The other implication also holds and gives a useful criterium to test if a theory is openly axiomatizable. For example, the $DeLO$ theory is not openly axiomatizable, because a finite substructure of its model is not its model.

Another interesting case is finite axiomatizability. A class of structures \emph{$K \subseteq M(L)$ is finitely axiomatizable}, if there is a finite theory $T$ such that $M(T) = K$. Similarly, \emph{a theory $T$ is finitely axiomatizable}, if $M(T)$ is finitely axiomatizable. 

There are actually classes of structures that are not axiomatizable at all. The compactness theorem gives us the following lemma:

\begin{lemma}
If a theory $T$ has for each $n \in \Nat$ at least $n$-element model, it has an infinite model.
\end{lemma}
\begin{proof}
For languages without equality this is true by the Lövenheim-Skolem theorem. In languages with equality, we create an extension $T'$ of $T$, such that $T' = T \cup \{c_i \neq c_j| i \neq j\}$ in a language with countably many new constant symbols $c_i$. Every finite part of $T'$ has a model according to the assumption and therefore $T'$ also has a model according to the compactness theorem. This model must be infinite, and its reduct to the original language is the desired infinite model of $T$.
\end{proof}

However, that means that if a theory has for each $n \in \Nat$ at least $n$-element model, the class of its finite models is not axiomatizable, e.g. the class of finite groups of finite fields are not axiomatizable in first-order logic.

\section{Incompleteness}

We will now show the limits of all formal logical systems. In the begging of the $20^{th}$ century, Hilbert proposed so called Hilbert's program, whose goal was to ground all existing mathematical theories to a finite, complete set of axioms. We will show that such a program is unattainable for key areas of mathematics (as shown by Gödel in 1931), more specifically, we will show that the arithmetic on natural numbers represented by the structure $\UNat = \struct{\Nat, S, +, \cdot, 0, \leq}$ is not recursively axiomatizable (this is a corollary of the Gödel's incompleteness theorem).

We will start by introducing a first approximation of $\Th(\UNat)$. The so called \emph{Robinson arithmetic} $Q$ has finitely many axioms:
\begin{multicols}{2}
\begin{enumerate}
  \item $S(x) \neq 0$
  \item $S(x) = S(y) \to x = y$
  \item $x + 0 = x$
  \item $x + S(y) = S(x+y)$
  \columnbreak
  \item $x \cdot 0 = 0$
  \item $x \cdot S(x) = x \cdot y + x$
  \item $x \neq 0 \to (\exists y)(x = S(y))$
  \item $x \leq y \lequiv (\exists z)(z +x = y)$
\end{enumerate}
\end{multicols}

The Robinson arithmetic $Q$ is rather weak -- it does not even prove the commutativity or associativity of $+,\cdot$ or the transitivity of $\leq$. On the other hand it suffices to prove for example existential sentences on numerals that are true in $\UNat$. For example, for $\varphi(x,y)$ in the form $(\exists z)(x+z=y)$ it is $Q \vdash \varphi(\underline{1}, \underline{2})$, where $\underline{1} = S(0)$ and $\underline{2} = S(S(0))$.

A stronger approximation of $\Th(\UNat)$ is an extension of the Robinson arithmetic called the \emph{Peano arithmetic} $PA$. It has all the axioms of Robinson arithmetic and the \emph{scheme of induction} -- for every formula $\varphi(x, \vec{y})$ of $L$ the axiom $$(\varphi(0, \vec{y}) \land (\forall x)(\varphi(x, \vec{y}) \to \varphi(S(x), \vec{y}))) \to (\forall x)\varphi(x, \vec{y})\,.$$

Peano arithmetic is a quite common approximation of $\Th(\UNat)$, it proves all the basic properties that are true in $\UNat$, but it is still incomplete, i.e. there are sentences that are true in $\UNat$ but not independent in $PA$. 

One could hope, that there is another, even stronger, extension of Robinson arithmetic such that it recursively axiomatizes $\Th(\UNat)$, but we will show that for every consistent recursively axiomatized extension $T$ of Robinson arithmetic there is a sentence true in $\UNat$ and unprovable in $T$ (this is the first Gödel's incompleteness theorem).

Before we get to the incompleteness theorem, we will show another rather surprising result -- there is no algorithm that can decide whether a given sentence is logically true. The proof of this proposition is based on the solution to the $10^{th}$ Hilbert's problem that asks to find an algorithm that determines in finitely many steps whether a given Diophantine equation is an arbitrary number of variables and with integer coefficients has an integer solution (\emph{Diophantine equations} are of form $p(x_1, \dots, x_n) = 0$ for some polynomial $p$ with integer coefficients). The solution to this problem was found in 1970, and the result is that the problem of existence of integer solutions to a given Diophantine equation with integer coefficients in algorithmically undecidable. (We will not prove it.) But that also means that there is no algorithm to determine for given polynomials $p(x_1, \dots, x_n), q(x_1, \dots, x_n)$ with natural coefficients whether $$\UNat \vDash (\exists x_1)\dots(\exists x_n)(p(x_1, \dots, x_n) = q(x_1, \dots, x_n))\,.$$

Assume now that there is an algorithm that can decide the logical truth of sentences. In particular, such an algorithm could decide the logical truth of the sentence $\psi \to \varphi$, where $\psi$ is the conjunction of closures of axioms of the Robinson arithmetic and $\varphi$ is the formula $(\exists x_1)\dots(\exists x_n)(p(x_1, \dots, x_n) = q(x_1, \dots, x_n))$ for some polynomials $p$ and $q$ with natural coefficients, i.e. the algorithm decides whether $\vDash \psi \to \varphi$, according to the completeness theorem, this is equivalent to $\vdash \psi \to \varphi$, which is again equivalent (by the deduction theorem) to $Q \vdash \varphi$. Because $Q$ proves existential sentences on numerals that are true in $\UNat$\sidenote{More precisely, we have for every existential formula $\varphi(x_1, \dots, x_n)$ in arithmetic $Q \vdash \varphi(x_1/\underline{a_1}, \dots, x_n/\underline{a_n})$ if and only if $\UNat \vDash \varphi[e(x_1/a_1, \dots, x_n/a_n)]$}, the algorithm also decides $\UNat \vDash \varphi$ which is not possible by the answer to the $10^{th}$ Hilbert's problem.

The proof of the Gödel's incompleteness theorem is based on so called arithmetization and on self-reference. The arithmetization shows that the syntactic notions of proofs can be expressed as formulas in arithmetic. We start by assigning numbers to all symbols of the language and then create a function, that assigns values to sequences (i.e. terms, formulas, and so on). In this way, all the finite objects (variables, terms, formulas, tableaux, proofs) can be assigned a natural number (code). All these codes can be computed recursively and are reversible (i.e. we can recursively compute the original object from the code). Let $\lceil \varphi \rceil$ denote the code assigned to formula $\varphi$ and let $\underline{\varphi}$ denote the numeral representing $\lceil \varphi \rceil$.

It $T$ has a recursive axiomatization then the predicate $\Prf_T \subseteq \Nat^2$ defined as $$\Prf_T(x, y) \Leftrightarrow y \text{ is a proof of } x$$ is recursive. We can easily create an algorithm that first decodes $x$ and $y$ and then checks if $y$ is indeed a tableau proof of $x$ -- axioms of $T$ are recursive, thus we can check if a given entry in tableau is an axiom of $T$, and the atomic tableaux can also be checked recursively). If, moreover, $T$ extends the Robinson arithmetic $Q$, the relation $\Prf_T$ can be represented by a formula $Prf_T$ such that for every $x, y \in \Nat$, $Q \vdash Prf_T(\underline{x}, \underline{y})$ if $\Prf_T(x,y)$ and $Q \vdash \neg Prf_T(\underline{x}, \underline{y})$ otherwise.

The formula $Prf_T(x,y)$ expresses that $y$ is a proof of $x$ in $T$. Therefore, the formula $(\exists y)Prf_T(x, y)$ expresses that $x$ is provable in $T$. If $T \vdash \varphi$ then $\UNat \vDash (\exists y)Prf_T(\underline{x}, y)$ and moreover $T \vdash (\exists y)Prf_T(\underline{x}, y)$.

The other important part of the proof of the incompleteness theorem is self-reference. A self-referential sentence can mention itself. For example, ``This sentence has 24 letters'' is an example of a self-reference. However, such direct self-reference is not always available in formal systems. On the other hand, direct reference to another sentence is often available, like in ``The following sentence has 32 letters `The following sentence has 32 letters' ''. However, it is not self-referential. We can use direct reference to create self-reference, like in ``The following sentence written once more and then once again between quotation marks has 116 letters `The following sentence written once more and then once again between quotation marks has 116 letters' ''.

Under some assumption, there is a self-referential formula $\psi$ that expresses ``this formula satisfies condition $\varphi$'' for any condition $\varphi$, as demonstrated by the next theorem.

\begin{theorem}[Fixed-point theorem]
Let $T$ be a consistent extension of Robinson arithmetic. For every formula $\varphi(x)$ there is a sentence $\psi$ such that $T \vdash \psi \lequiv \varphi(\underline{\psi})$.
\end{theorem}
\begin{proof}
We will only show the idea of the proof. Consider the doubling function $d$ such that for every formula $\chi(x)$: $d(\lceil \chi(x) \rceil) = \lceil \chi(\underline{\chi (x)}) \rceil$. It can be shown that such a formula is expressible in $T$, assume it is by a term called also $d$.

Then, for every formula $\chi(x)$ of the language of the theory $T$ holds that $$T \vdash d(\underline{\chi(x)}) = \underline{\chi(\underline{\chi(x)})}\,.$$ We can now take $\psi$ as $\varphi(d(\underline{\varphi(d(x))}))$. It suffices to verify that $T \vdash d(\underline{\varphi(d(x))}) = \underline{\psi}$. It follows from the above equation for $\chi(x)$ being $\varphi(d(x))$ since in this case $$T \vdash d(\underline{\varphi(d(x))}) = \underline{\varphi(d(\underline{\varphi(d(x))}))}\,.$$
\end{proof}

The consequence of the previous theorem is that we cannot define truth in any consistent extension of Robinson arithmetic. We say that a formula $\tau(x)$ \emph{defines truth} in theory $T$ of arithmetical language if for every sentence $\varphi$ it holds $T \vdash \varphi \lequiv \tau(\underline{\varphi})$. In the proof we will use the principle of self-reference and the liar's paradox. The sentence $\varphi$ expresses ``This sentence is not true in $T$''. 

\begin{theorem}[Undefinability of truth]
Let $T$ be a consistent extension of Robinson arithmetic. Then $T$ has no definition of truth. 
\end{theorem}
\begin{proof}
Assume for contradiction that there is a formula $\tau(x)$ that defines truth in $T$. By the fixed-point theorem for $\neg \tau(x)$ there is a sentence $\varphi$, such that $T \vdash \varphi \lequiv \neg \tau(\underline{\varphi})$. As $\tau$ defines truth this means that $T \vdash \varphi \lequiv \neg \varphi$ which is not possible in a consistent theory.
\end{proof}

We can finally prove the first Gödel's incompleteness theorem, that shows that whatever we do, there is not a consistent recursively axiomatizable extension of Robinson arithmetic that is equivalent to $\Th(\UNat)$.

\begin{theorem}[First Gödel's Incompleteness Theorem]
For every consistent axiomatizable extension $T$ of Robinson arithmetic there is a sentence true in $\UNat$ and unprovable in $T$.
\end{theorem}
\begin{proof}
Let $\varphi(x)$ be $\neg (\exists y)Prf_T(x, y)$, i.e. it says $x$ is not provable in $T$. By the fixed point theorem there is a sentence $\psi_T$, such that $T \vdash \psi_T \lequiv \neg (\exists y)Prf_T(\underline{\psi_T}, y)$, such a $\psi_T$ is equivalent to a formula that expresses that $\psi_T$ is not provable in $T$, and the equivalence holds both in $T$ and $\UNat$. 

First, we show that $\psi_T$ is not provable in $T$. If it is, i.e. if $T \vdash \psi_T$, then $\psi_T$ is contradictory in $\UNat$, therefore $\UNat \vDash (\exists y)Prf_T(\underline{\psi_T}, y)$ and moreover $T \vdash (\exists y)Prf_T(\underline{\psi_T}, y)$. Thus also $T \vdash \neg \psi_T$, which is impossible since $T$ is consistent.

Now we show $\psi_T$ is true in $\UNat$. If not, i.e. $\UNat \vDash \neg \psi_T$, then $\UNat \vDash (\exists y)Prf_T(\underline{\psi_T}, y)$, therefore $T \vdash \psi_T$ which is not possible as shown in the previous paragraph.
\end{proof}

If we additionally add the assumption that $\UNat \vDash T$, then $T$ is incomplete, because if it were complete then $T \vdash \neg \psi_T$ and thus $\UNat \vDash \neg \psi_T$, which contradicts $\UNat \vDash \psi_T$. This also shows that $\Th(\UNat)$ is not recursively axiomatizable, because $\Th(\UNat)$ is a consistent extension of Robinson arithmetic and has a model $\UNat$. Suppose $\Th(\UNat)$ is recursively axiomatizable. However, by the previous discussion, $\Th(\UNat)$ is incomplete, which is a contradiction with $\Th(\UNat)$ being complete (as $\Th(\cA)$ is complete for any structure $\cA$).

The Gödel's theorem was strengthened by Rosser. The Rosser's theorem states that every consistent recursively axiomatized extension $T$ of Robinsons arithmetic has an independent sentence and is thus incomplete. Thus the assumption $\UNat \vDash T$ was not needed above.

The second Gödel's incompleteness theorem shows that the consistency of a theory cannot be proved in the theory itself. More formally, let $Con_T$ denote the sentence $\neg (\exists y)Prf_T(\underline{0 = 1}, y)$. We have $\UNat \vDash Con_T \Leftrightarrow T \nvdash 0 = \underline{1}$. Thus, $Con_T$ represents ``$T$ is consistent''.

\begin{theorem}[Second Gödel's Incompleteness Theorem]
For every consistent recursively axiomatized extension $T$ of Peano arithmetic it holds that $Con_T$ is unprovable in $T$.
\end{theorem}
\begin{proof}
Let $\psi_T$ be the sentence from the proof of the first Gödel's incompleteness theorem, i.e. ``this is not provable in $T$''. We have shown in the proof that if $T$ is consistent, then $\psi_T$ is not provable in $T$, i.e. $Con_T \to \psi_T$. If $T$ is an extension of Peano arithmetic the proof can be formalized within the theory of $T$ itself. Hence $T\vdash Con_T \to \psi_T$. Since $T$ is consistent, we have that $T\nvdash \psi_T$, therefore if follows that $T \nvdash Con_T$. 
\end{proof}

Because the Peano arithmetic is recursively axiomatizable extension of itself and $Con_T$ is not provable in $PA$, there must be a model $\cA$ of the Peano arithmetic such that $\cA \vDash (\exists y)Prf_{PA}(\underline{0 = 1}, y)$. The model $\cA$ must be a so called non-standard model of arithmetics and the witness for the existence in the sentence must be some non-standard element of the model (other than a value of a numeral).

Similarly, there is also a consistent recursively axiomatizable extension $T$ of Peano arithmetic such that $T \vdash \neg Con_T$. If we choose $T = PA \cup \{\neg Con_{PA}\}$, then $T$ is consistent, because $T \nvdash Con_{PA}$. Moreover, $T \vdash \neg Con_{PA}$, i.e. $T$ proves inconsistency of $PA \subseteq T$, and thus also $T \vdash \neg Con_T$. Again, all models of $T$ must be non-standard, i.e. $\UNat$ is not a model of $T$.

The second Gödel's incompleteness theorem actually states that a recursively axiomatized extension of Peano arithmetic cannot prove its consistency, unless it is inconsistent. Since the Zermelo-Frankel set theory (ZFC) is strong enough to define the required arithmetic notions, it also means that if ZFC is consistent then $Con_{ZFC}$ is unprovable in ZFC. From a more philosophical point of view, this means that we actually cannot prove if modern mathematics (based on ZFC) is consistent, unless it is inconsistent.

\bigskip

This concludes the last part of the introductory lecture to logic. The goal of this part was to introduce more advanced parts of logic and provide some strong results that can be used to analyze a given theory -- we provided useful criteria for completeness and decidability of theories. We have also shown the limits of the formal system, mostly represented by the Gödel's incompleteness theorem and by the existence of classes of structures that are not axiomatizable in first-order logic. 